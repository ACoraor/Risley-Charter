\documentclass[12pt]{article}
\topmargin=0in
\oddsidemargin=0in
\evensidemargin=0in
\textwidth=6.5in
\marginparwidth=0.5in
\renewcommand{\thesection}{\Roman{section}} 
\usepackage{hyperref}
\usepackage{enumerate}
\begin{document}
\centerline{\huge{Risley Charter and By-Laws}}
\centerline{\textsc{Current as of 2 Nov 2014}}
\hrulefill
\section*{Risley Charter}
\section{Name}
\label{sec:name}
The name of this organization shall be Risley Residential College for the Creative and Performing Arts. \textsc{\footnotesize{(Spring 1987)}}
\section{Purpose}
The purpose of Risley Residential College is to promote the creative and performing arts in the Cornell and Ithaca communities.
\section{Membership and Fees}
\begin{enumerate}[A.]
\item There will be two categories of Risley membership: "In-House" and "Out-Of-House." Risley membership will be by application only and will be open to any interested person. Risley members may be residents (in-house) or non-residents (out-of-house). In-house Risley members must be full-time, registered students at Cornell University, and must also apply for residency. \textsc{\footnotesize{(rev. Fall 2002, Spring 2010)}}
\item All Risley members must pay an activity fee. The amount of the activity fee shall be approved by referendum among the membership. The results of such referendum must be approved by the parent organization specified by the By- Laws and can take effect no sooner than the beginning of the academic year following its proposal. \textsc{\footnotesize{(rev. Spring 2002)}}
\item Any Cornell-affiliated person who so wishes may, after successful application to Risley, pay the Risley activity fee and become an out-of-house member. Out-ofhouse members will have full access to the Risley shops on the same basis as inhouse Risley members. Fees collected from Risley out-of-house members will be applied to the general fund for programs and maintenance. This fund will be administered by the Risley Kommittee. Out-of-house Risley members retain full membership rights. \textsc{\footnotesize{(rev. Spring 1996, Spring 2010)}}
\end{enumerate}
\section{The Risley Shops}
\begin{enumerate}[A.]
\item Shop Managers
\begin{enumerate}[1.]
\item All Risley Shops or Shop-groups must have at least one Manager who currently resides in Risley. The duties of Shop Managers shall be determined by the Risley Kommittee, with the RHD and the Shop Coordinator, and shall include at least the following: (rev. Spring 2010)
\begin{enumerate}[a.]
\item Maintain the Shop facilities ensuring they are orderly and safe;
\item Maintain records of the Shop's inventory;
\item Run semesterly workshops on the shop resources;
\item Publicize shop resources.
\end{enumerate}
\item Shop/Shop-group Managers will be appointed by Kommittee in conjunction with the RHD.
\end{enumerate}
\item Shop Coordinator
\begin{enumerate}[1.]
\item A Shop Coordinator, whose duties shall be determined by the Risley Kommittee and the RHD. These shall include at least the following: (Spring 2010)
\begin{enumerate}[a.]
\item Attending all Kommittee meetings, and delivering the weekly "Shop Report" to inform the Kommittee of the status of the shops and of any shop events;
\item Records the details of every allocation for the shops by Kommittee;
\item Maintains records of the shop inventories (performed by shop managers);
\item Maintains and distributes all shop keys;
\item Hold regular meetings with the Shop Managers;
\item Evaluates shop manager performance.
\end{enumerate}
\item Kommittee may fiscally compensate the Shop Coordinator at its option, with pay determined and allocated at the start of each semester.
\end{enumerate}
\item Each shop may, at the discretion of the manager(s), establish a group of shop associates. Shop associates are non-members who wish to use that shop only. Shop Associates would pay a set fee for the use of that shop only. Risley memberswill not be assessed shop fees in addition to the Risley Activity Fee. Shop fees will be set by the manager(s) and Kommittee.
\end{enumerate}
\section{The Risley Kommittee: Name and Jurisdiction}
\begin{enumerate}[A.]
\item The phrase \textsc{the Risley Kommittee} or \textsc{Kommittee} shall mean those people holding the Greco-Roman\footnote{Blame Joe Otchin (Spring 2004) for this.} positions defined in Article VI. A-C of the Charter (Spring 1985, rev. Spring 1987, rev. Fall 2002)
\item Kommittee shall have jurisdiction over all specific requests for spacetime and money, but shall act in accordance with spaaaaaace use and budget policies developed by Risley members. Questions of spacetime and fiscal policy, and all other decisions, aside from those few exceptions specified in the Charter or By-Laws, are open to other Risley members to be determined via referenda. (rev. Fall 2002)
\section{The Risley Kommittee: Membership and Institution}
\begin{enumerate}[A.] 
\item The elected Kommittee members shall be:
\begin{enumerate}[1.]
\item \textbf{Members-At-Large.} Eight At-Large Representatives.
\item \textbf{Minister of Letters.} A Minister of Letters (AKA Secretary) shall act as recording secretary, keeping minutes of all meetings, and shall post minutes within two days of the next meeting, and at least 15 minutes after the meeting is adjourned. As soon as the meeting is adjourned, the minister will ask Kommittee if there is anything that they need to add to the minutes.
\begin{enumerate}[a.]
\item The Secretary shall also be responsible for counting hands in the case of a hand vote. (Spring 2011)
\item All records kept by the recording secretary shall be given to the Archivist as soon as they are no longer in use. (Spring 1999)
\end{enumerate}
\item textbf{Minister of Propaganda.} The Minister of Propaganda (AKA Propagandist) shall adequately publicize all Kommittee events (eating contests, referenda, meetings, etc) as soon as is practical. They shall be in charge of any votes that involve ballots, including eating contests and referenda. (Fall 2003, rev. Spring 2006, rev. Spring 2011)
\item The Kommittee Chair, AKA "¡El Presidente For Life!" \footnote{Blame Dante Heifets (Spring 2002) for this.} The Chair shall be elected by Kommittee from the directly-elected or endured members at the last meeting of each semester. All nominees shall be required to state their intent to run at the second to last Kommittee of the semester. The newly elected chair shall not take office until the first Kommittee of the next semester, however, they shall be responsible for any Kommittee matters dealt with over summer or winter break. The Chair shall be responsible for the smooth running of all Kommittee meetings. The Chair shall restate all motions before they are put to a vote. (rev. Spring 2001)
\end{enumerate}
\item The appointed Kommittee members shall be:
\begin{enumerate}[1.]
\item \textbf{Resident Advisors.} Resident Advisors (RAs), whose duties shall be determined by the parent organization specified in the By-Laws in conjunction with the Risley Kommittee. These shall include at least the following:
\begin{enumerate}[a.]
\item shall be responsible for taking necessary security precaution for Risley property;
\item shall be responsible for taking year end inventory.
\end{enumerate}
\item \textbf{Residence Hall Director.} Residence Hall Director (RHD), whose duties shall be determined by the parent organization specified in the By-Laws in the spring prior to the academic year in which s/he will serve. (Fall 2003) a. In the absence of Kommittee, the Residence Hall Director shall be responsible for making any pressing decisions that are in the best interest of Risley and the Risley community. b. Between Kommittee meetings, the RHD or any person(s) they may designate shall have the authority to make any pressing decisions they consider in the best interest of Risley. The RHD shall inform Kommittee of any such decisions at the next Kommittee meeting. (Spring 1999, rev. Fall 2002, Spring 2010)

\item \textbf{Grand Vizier.} A Grand Vizier (aka Risley Accountant), whose duties shall be determined by the Risley Kommittee and shall include at least the following: a. Attending all Kommittee meetings, and delivering the weekly "Budget Report" to inform the Kommittee of Risley's fiscal situation; b. Record the details of every fiscal allocation by Kommittee; c. Collect and record the details of every receipt pertaining to a fiscal allocation by Kommittee; d. Make available the details and summary of Risley's budget to any interested party; e. Record the details of every allocation of spaaaaaace within Risley; f. Make available the details and summary of the available spaaaaaace within Risley to any interested party; g. Act as the primary point of contact between any outside group seeking or desiring to seek Risley resources; h. Record the contact information of any outside group that is allocated Risley resources.
Kommittee may fiscally compensate the Grand Vizier at its option, with pay determined and allocated at the start of each semester. (Fall 2003)
\item \textbf{Artist in Residence.} The Artist in Residence(s) is/are required to attend Kommittee meetings and present reports at the beginning of each meeting.
\item \textbf{Kommittee Mime.} A Kommittee member may be designated Kommittee Mime, to be detonated in times of despair. \footnote{I'm not entirely sure who is to blame for this.}
\item \textbf{Kommittee Barber.} The Kommittee Barber shall shave all and only Kommittee members who do not shave themselves. \footnote{Blame Joe Otchin (Spring 2004) and David Schoonover (Fall 2006 - should have been Spring 2005) for this.}
\item No appointed member may run for or hold an elected Kommittee position. (Spring 2011)
\end{enumerate}
\item Members-By-Endurance. Any Risley member may at any time declare his or her intention to become a Member-By-Endurance of Kommittee. They are then required to attend two consecutive full regular Kommittee meetings. At the third meeting they become a Member-By-Endurance with full voting rights. If a person has been endured previously, they become re-endured by the second attended meeting, instead of the third. (Fall 2003, rev. Spring 2010, rev. Spring 2012)
\begin{enumerate}[1.]
\item Endured members may have absences excused by the Chair, starting at the first meeting of the new Kommittee. This excuse may be granted after the absence has occurred. Two unexcused absences result in loss of Kommittee member status. Excused absence rules apply during the probationary period. (rev. Spring 2012)
\item Risley members who wish to become Kommittee members can start enduring for Kommittee at any time.
\item Endured membership does continue from year to year, providing that the endured member has complied with attendance requirements and is still otherwise eligible for Kommittee membership. (moved Spring 2006)
\end{enumerate}
\item All members of Kommittee must be members of Risley. (rev. Fall 2007)

\item (Spring 2005, deleted Spring 2011)

\item Excused Absences. Absences from Kommittee meetings may be granted by the Kommittee Chair only; therefore, the Chair shall also be responsible for keeping attendance of regular Kommittee members and overseeing the attendance of declared Representatives-by-Endurance. Under normal circumstances, permission to miss all or part of a Kommittee meeting shall not be granted once the meeting has begun. At the beginning of each meeting, the names of all Kommittee members who have been given leave to miss the meeting shall be announced and recorded in the minutes. (Spring 1986, rev. Spring 2010, rev. Spring 2011, rev. Spring 2012)

\item In-House Conflicts. Members involved in temporary Risley events which conflict in time with Kommittee meetings will be excused by Kommittee ahead of time. These absences will not affect the member's attendance record. The membership of Kommittee will be temporarily reduced by the number of persons excused under this provision, so that the quorum will be lowered.

\item Quorum. No Kommittee business may be done without a quorum present or anywhere but at a Kommittee meeting except as noted in the Charter or By-Laws. A quorum shall be defined as a number of Kommittee members exceeding one-half the total number of elected and appointed Kommittee members (not to include members by endurance). (rev. Fall 2002, rev. Spring 2006)

\item Impeachment. At any time, any member of Kommittee may call for a Vote of No Confidence as specified in the Parliamentary Rules. (Fall 2003)
\end{enumerate}
\end{enumerate}
\section{Rules Governing Kommittee Actions}
\begin{enumerate}[A.]
\item Rules for Kommittee meetings shall be decided by the Risley membership. The chair of the Risley Kommittee shall preside over any Kommittee meeting. If, for a single meeting, no Chair is available, then the Chair shall appoint a replacement from the elected members to preside over the meeting. (rev. Fall 2002, rev. Fall 2006, rev. Spring 2010)
\item Regular Kommittee meetings will be held each Sunday followed by a Monday on which classes are scheduled during the academic year.
\item Proxy votes will not be accepted at Kommittee, although votes may be registered with the Chair in advance.
\item Kommittee may hold special meetings, including emergency meetings.
\begin{enumerate}[1.]
\item Special Kommittee meetings shall be called by the Chair, the RAs, or the RHD. The agenda for such meetings shall be finalized and posted by the caller before the meeting. Absences from special Kommittee meetings may, in mitigating circumstances, be granted by the Chair with no impact on a member's attendance record.
\item Special general meetings may be called by the RHD, an RA, or Kommittee.
\item Emergency meetings: The quorum may not be lowered for any emergency meetings. An adequate attempt must be made to inform all Kommittee members about emergency meetings. Absence from an emergency Kommittee meeting will not affect a member's attendance record.
\end{enumerate}
\item A majority of Kommittee members present may overrule a decision of the Chair.
\item Immediately after a Kommittee meeting the Chair shall post the agenda for the next meeting. The Chair shall decide the order of the agenda. New items may be added at the time of the meeting.
\item Switzerland Clause. During voting, when the House is Divided, the RHD and RAs have the option from obstaining from voting. (Spring 2010)
\item No Kommittee member may be forced to leave the room and thereby be absent for a discussion or vote. (Spring 2011)
\item Kommittee proceedings are public. Nothing may be stricken from the minutes to protect confidentiality. (Spring 2011)
\end{enumerate}
\section{Subcommittees}
\begin{enumerate}[A.]
\item Kommittee may authorize new subcommittees to accomplish specific tasks (ex. Staff Selection subcommittee) or activate dormant subcommittees to accomplish recurring tasks (ex. Theatre Subcommittee, AIR Selection Subcommittee) by a 3/4 vote. These subcommittees remain active until their task is complete.
\item All subcommittees shall have a charter outlining their powers and responsibilities independent of the full Risley Kommittee.
\begin{enumerate}[1.]
\item No subcommittee may have powers that exceed those of the Kommittee.
\item All active subcommittees shall make a report to the Kommittee at each regular Kommittee meeting.
\item Kommittee and only Kommittee may amend any Subcommittee charter by a 2/3 vote.
\end{enumerate}
\item The full Risley Kommittee may at any time disband a standing subcommittee and assume its powers and responsibilities by a 3/4 vote.
\end{enumerate}
\section{Elections and Transitions}
\begin{enumerate}[A.]
\item Elections for Risley Kommittee positions shall be held once each semester. All elected members from the previous semester ("Lame Ducks") continue until the next election. Further regulations governing Kommittee elections will be found in the By-Laws.
\item Two-Headed Monster Rule. All Kommittee positions (except those of Membersat- Large and staff, but including the position of Chair) may be held jointly by a team of two people provided that those two people ran as a team for the position. Such a team shall have two votes at Kommittee, and each member shall be counted separately in establishing a quorum. (rev. Spring 1989, moved Fall 2002)
\item Dates and procedures for the elections of all elected members shall be found in the By-Laws. If no alternate wishes to hold the office, then any Member-By- Endurance may be appointed by a 2/3 vote, provided that person wishes to hold the position. (rev., moved Fall 2002)
\item The New Blood Rule. The fall semester membership of Kommittee (excluding Members-By-Endurance) shall contain at least two New-to-Risley ("new bloods"). If no two new bloods receive enough votes to be elected in their own right, the Member-at-Large seats shall all be filled as follows:
\begin{enumerate}[1.]
\item If only one new blood receives enough votes to be elected, the other new blood Member-at-Large candidate with the most votes shall be elected as Member-at-Large.
\item If no new bloods receive enough votes to be elected, the two new blood Member-at-Large candidates with the most votes shall be elected as Member-at-Large.
\item The remaining six seats shall be filled by the other candidates with the greatest number of votes.
\end{enumerate}
\item (rev., moved Fall 2002, rev. Fall 2003, rev. Fall 2007)
\item If, at the end of the semester, any elected Kommittee member knows they will not be able to continue their duties as lame duck at the beginning of the coming semester (e.g. due to graduation, study abroad, or general abandonment of Risley), their position shall be included in the Special Election at the end of the semester. The newly elected person shall carry out the rest of their predecessor’ s term, and the position shall be open again for the next election. In the event that a position is left vacant unexpectedly, and must be filled immediately, the GV shall appoint an interim until a Special Election can be held. (Spring 2010, rev. Spring 2011)
\end{enumerate}
\section{Selection of Residence Hall Staff}
\begin{enumerate}[A.]
\item Ultimate hiring/appointing authority of all residence hall staff (RAs, RHD) shall be held by the parent organization specified in the By-Laws. The Risley Staff Selection sub-committee must proceed as set forth by the parent organization's policies and procedures. The RHD shall still be the arbiter of any/all in-house selection processes in conjunction with the Risley Kommittee / Staff Selection Sub-Committee.
\item In accordance with the parent organization of specified in the By-Laws, residence staff must meet eligibility criteria set forth by said organization and Risley applicants must also comply.
\item The RHD also has ultimate responsibility to ensure that all documents and policies regarding staff selection for a given year will be available upon request. (rev. Spring 1995, rev. Fall 2005)
\end{enumerate}
\section{Artist in Residence Selection}
\begin{enumerate}[A.]
\item No more than two Artists in Residence shall be appointed for each semester. Each Artist in Residence ("AIR") shall reside in a Risley Guest Suite.
\item For additional details, please refer to the Artist in Residence Subcommittee Charter (Spring 2007, rev. Spring 2010)
\end{enumerate}
\section{Referenda}
\begin{enumerate}[A.]
\item A referendum shall be initiated by Kommittee as stated in the By-Laws or by submitting to the Chair a petition containing the signatures of thirty Risley members. Procedures for the running of referenda shall be found in the By-Laws.
\item Kommittee shall not act to change policy previously established by referendum. They may call for a new referendum. (moved Fall 2002)
\item Intention to propose a motion to reconsider, or to overturn by referendum, any issue once voted on (by Kommittee or at referendum) must be announced no later than the second regular Kommittee meeting following the original decision, unless circumstances have demonstrably changed or new facts come to light. (moved Fall 2002)
\item The Risley Kommittee shall have the authority to delay voting on a referendum by up to two weeks if the ordinary voting day would fall just before, or during, a vacation or other period when many Risleyites would be out of town. (moved Fall 2002)
\end{enumerate}
\section{Suspension}
\begin{enumerate}[A.]
\item No part of the Charter may be suspended unless specifically permitted by the Charter itself.
\end{enumerate}
\section{Amendments to the Charter}
\begin{enumerate}[A.]
\item Amendments to the Charter shall be passed in accordance with the provisions established in section XV. By-Laws.
\item Amendments to Charter may be made at any time, but there will be a meeting of Kommittee designated, once a year, to deal with amendments to Charter. This meeting will be called Charter Review.
\item If a proposed amendment to Charter is deemed controversial, and it goes to referendum, there must be at least two choices, even if one of them is simply a reiteration of the current policy.(Spring 2011)
\item All potential amendments to Charter must be presented in writing to Kommittee prior to voting.  This does not apply to those sent to referendum, for which written options must be presented to the whole of Risley, as per XII. C. (Fall 2014)
\end{enumerate}
\section{By-Laws}
\begin{enumerate}[A.]
\item By-Laws will be placed into four categories:
\begin{enumerate}[1.]
\item Parliamentary Rules relate to the functioning of Kommittee, to maintenance of the Charter, or to the disposition of motions at Kommittee meetings. These rules may be created, amended, or deleted by a 3/4 vote, unless otherwise specified in the rule itself.
\item Certain Policies may be created, amended or deleted only by referendum. These policies are: I. Name, II. Purpose, X. Selection of Residence Hall Staff, XI. Artist in Residence Selection, XIII. Suspension (Spring 2011, rev. Spring 2012)
\item General Policies may be created, amended, or deleted by a 2/3 vote. Any General Policy may become a Restricted Policy through referendum.
\item Policies governing Risley's relationship with specific outside groups ("OG Policies") are to be placed in the Outside Groups Appendix; these rules may be created, amended, or deleted by a majority vote.
\end{enumerate}
\item (rev. Fall 2006)
\item In the case of a conflict between elements to this document, policies are to be considered in the following order of precedence: Charter overrules all other elements; Restricted Policies overrule Parliamentary rules and General Policies; Parliamentary rules overrule General Policies.
\item A permanent record of the By-Laws in force shall be kept appended to the Charter and divided according to the proper classification, as outlined above. Each by-law adopted shall be appended along with the date of its adoption.
\item Furthermore, once each academic year, at a time to be determined as a Parliamentary Rule, the Charter and By-Laws shall be reviewed so as to keep the provisions and mechanisms of the Charter and By-Laws in the remembrance of all and to add or delete such provisions or rules as Kommittee or College sees fit.
\end{enumerate}
\section*{Parliamentary Rules}
\begin{enumerate}[1.]
\item \textbf{Regular Meetings.} Regular Kommittee meetings shall be held every Sunday followed by a Monday on which classes are scheduled during the academic year at 9:00 p.m.. This time may be changed by a majority vote of the full Kommittee membership. (rev. Fall 2007)
\item \textbf{Agenda Items.} Items on the agenda shall be signed or initialed by the person posting them, or who will present them to Kommittee. (Fall 1981, rev. Spring 1999)
\item \textbf{Charter Review.} At the third meeting of the new Kommittee for the spring semester, it will be announced that a review of the Charter and By-Laws will take place in two weeks. Copies of the Charter and By-Laws shall be made available to the general membership. At the fifth meeting of the new Kommittee for the spring semester, the Charter and By-Laws shall be reviewed. (Fall 1981, rev. Fall 1983)
\item \textbf{Calling the Question.} All motions to call the question must be passed by a 2/3 vote. Any Risley member in the room may vote. (Spring 1985, rev. Fall 2006)
\item \textbf{Tabling.} All motions to table a motion must be passed by a 2/3 vote. (Spring 1985, rev. Fall 2006)
\item \textbf{Motions to table on allocations over \$200 require only 2/5 vote.} (Fall 2004, rev. Fall 2006)
\item \textbf{Abstentions.} Abstentions shall be noted. (Fall 1985, rev. Spring 1999)
\item \textbf{Controversial Motions.} Controversial motions are those which strongly concern the whole of the Risley community, or those on which there exist strongly held and significantly opposing viewpoints. Motions of a possibly controversial nature shall be made known before Kommittee meetings on the main bulletin board and otherwise advertised to the building. If a motion is not posted for 48 hours beforehand, the motion may be determined controversial by a simple majority vote of present members of Kommittee. Any member of Kommittee may motion to deem a motion controversial. (Rev. Fall 2014) \footnote{Blame Christian Brickhouse (Fall 2014) for this.}
\begin{enumerate}[A.]
\item \textbf{Tabling a controversial motion.} A motion to table a controversial motion requires a simple majority to pass. During the time between tabling and the next Kommittee, the tabled motion must be advertised to the Risley community. Tabling should occur when parties need time to properly consider the motion, concerned parties are not present, or the motion is in too rough a form to be effectively revised by debate. (Rev. Fall 2014)
\end{enumerate}
\item \textbf{Absences.} Members who miss any part of a meeting will be considered absent, at the discretion of the Chair. (Fall 1988)
\item \textbf{Majority default.} All decisions not otherwise specified in this document shall be made by majority vote.
Dividing the House. After the vote on a main motion is taken, a member of Kommittee may immediately request a division of the house. If the motion is seconded and passes by a majority, a vote is taken in which all members who voted yes or no are required to vote in the same manner. Those who abstained in the original vote mow must vote for or against the motion. No abstentions are permitted during a division of the house. A division of the house may change the original outcome of the vote from passage to failure, or vice versa. (Fall 2002, rev. Fall 2006)
\item \textbf{Allocation of Spacetime.} Motions to allocate spacetime (not explicitly mentioned elsewhere in the Charter or By-Laws (e.g. the Great Hall)) must be passed by majority vote. (Fall 2002, rev. Fall 2006)
\item \textbf{Allocation of Small Funds.} Motions to allocate funds less than \$75 must be passed by majority vote. (Fall 2002, rev. Fall 2006)
\item \textbf{Allocation of Large Funds.} Motions to allocate funds of at least \$75 must be passed by a two-thirds vote. (Fall 2002, rev. Fall 2006)
\item \textbf{Vote of No Confidence.} A motion for a Vote of No Confidence regarding any specific Kommittee member will cause a 3/4 vote of Kommittee members present, as per the election procedure for a special election. If this vote succeeds, the Kommittee member in question is removed from his or her Kommittee office.
\item \textbf{Deallocations.} Motions to revoke an allocation (deallocate) require the same margin to pass as the original motion. (Fall 2007)
\end{enumerate}
\section*{Restricted Policies}
\begin{enumerate}[1.]
\item \textbf{Quiet Hours.} These policies apply inside the building as well as within the areas in the close vicinity of the building. The hours after 11pm Sunday-Thursday nights, after 1am Friday and Saturday nights, before 11am Saturday and Sunday mornings and before 10am on weekdays shall be designated as quiet hours in Risley. No excessive noise will be permitted after quiet hours. Extensions may be made for Risley events by a 2/3 vote. Any Risley member present in the room may vote. Quiet hours shall be extended until 3am for Risley performances of Rocky Horror Picture Show, performed by the Denton Drama Troupe. (rev. Spring 1998, rev. Spring 1999) 
\begin{enumerate}[A.]
\item Noise is defined as a sound level high enough to disturb a resident in a closed room.
\item During the daytime and evening hours, residents should follow the following procedure in the event of a noise infraction:
\begin{enumerate}[a.]
\item Contact the noisemaker, and negotiate a solution; if the doors and windows of both the plaintiff's and the noisemaker's room are not closed, try this first.
\item Should negotiating a solution fail, contact a staff member. 
\item Should both of the above fail, with no hope of solution, contact Cornell University Police.
\end{enumerate}
\item During quiet hours, Risley will follow a policy favoring the plaintiff; that is, all requests for quiet during quiet hours will be heeded. (rev. Spring 1988) 
\item At the end of each semester, Risley will observe a period of 21 hour quiet hours. They will begin at the time normal quiet hours would begin on the last day of scheduled classes. They shall continue until the end of the exam period. Kommittee shall decide when the three hour break from quiet hours shall be. If Kommittee does not set a different period, 5:00 - 8:00pm shall be standard. (Spring 1998)
\end{enumerate}
\item \textbf{Risley Library.} Hours between 7pm and 2am on Sunday through Thursday evenings will be designated as absolute silence hours in the Risley library. This rule shall not be applied if the library is otherwise allocated. Noise will be allowed during these hours only if no one objects. (Spring 1985, rev. Spring 1993, rev. Spring 1999)
\item \textbf{Outside Groups.} Whereas Risley Residential College was formed for the purpose of promoting the creative and performing arts, outside groups wishing to use Risley facilities (especially the Great Hall) would be required to have some basis in the creative and performing arts (exceptions specified below).
\begin{enumerate}[A.]
\item General
\begin{enumerate}[a.]
\item Risley Kommittee may make exceptions to the Outside Groups policy with a 2/3 vote. 
\item This policy does not affect allocations for the use of the Great Hall by groups of Risleyites. Also, this will not conflict with General Policy 26. 
\end{enumerate}
\item Allocations 
\begin{enumerate}[a.]
\item A 2/3 vote shall be required to allocate the Great Hall (Dining Hall) for the use of groups from outside of Risley.
\item In order to insure Risley property and compliance with staff and Kommittee guidelines, outside groups can be required, at the discretion of Kommittee, to make a security deposit for the use of Risley space. The amount of this deposit shall be \$100.00 for all rooms except the Great Hall, for which the amount of the security deposit shall be \$200.00. Use of any Risley space governed by Kommittee by an outside group can be subject to a \$50 rental fee, at the discretion of Kommittee. Any group that uses Risley space on an ongoing basis must have at least one Risley Membership. The Risley RHD (or his/her designee) shall oversee the deposit and form. (rev. Spring 2011, rev. Spring 2012) 
\item All outside groups wishing to hold an event in the Great Hall that is not directly related to the creative and performing arts must submit a one page, typewritten letter stating the exact reason they wish to use Risley's Great Hall above all other spaces on campus or in the Ithaca community. At least five copies of the letter will be circulated during the meeting at which the request for allocation of the Great Hall is presented, and the Kommittee Chairperson will read aloud the letter before discussion. 
\end{enumerate}
\item Equipment 
\begin{enumerate}[a.] 
\item Outside groups wishing to use Risley's equipment shall be assessed a usage fee. This fee shall be determined by the manager responsible for equipment. In the event that there is no manager responsible for the equipment, the outside group will be assessed a \$50.00 rental fee. All allocations of Risley equipment must be approved by Risley Kommittee, the RHD, or their designee. (rev. Spring 2010) 
\end{enumerate}
\item Sponsorship 
\begin{enumerate}[a.] 
\item A group will be assessed an inside group if its membership is composed of at least 50\% Risleyites, or if it was founded in Risley. The designation of a specific group can be challenged at any time, and changed by a majority vote of Kommittee. (rev. Spring 2010) 
\item Risley may co-sponsor events by a 2/3 vote. This means that deposit guidelines may be lowered or abolished by Kommittee. (moved Sprin 1992) 
\end{enumerate}
\item Restricted Groups 
\begin{enumerate}[a.]
\item Groups in Poor Standing--those who are not allowed to use the Dining Hall or other Risley space due to previous misuse--shall remain in Poor Standing for a period of 3 years. A representative of the group in question shall be contacted at least one week before a vote to place them on the list to allow them to make their position known. 
\item Groups whose security deposit or portions thereof are withheld will automatically be demoted to Poor Standing. Exceptions may be made by a 2/3 vote within two Kommittee meetings after the event. 
\item Groups may also be demoted to Poor Standing by a 2/3 vote.
\end{enumerate}
\end{enumerate}
\item \textbf{The Rocky Horror Picture Show.} (rev. Fall 2006, rev. Spring 2010) 
\begin{enumerate}[A.]
\item Each year, money will be allocated for the rental fees and rights to show the movie, costumes, props and set (if necessary). 
\item In the event that the performance of “The Rocky Horror Picture Show” loses money or does not make at least their allocation amount in profits, the Denton Drama Troupe will attempt to request funding from the SAFC in light of the disaster.
\item In the event that the Director of the Denton Drama Troupe is not a Risleyite, it is expected that the Denton Drama Troupe will select a current Risley resident from the cast to serve as liaison to the staff (hereafter known as the Denton Bitch) and who will ultimately be held accountable for the following: 
\begin{enumerate}[a.]
\item UUPs and other forms required by Cornell 
\item Fire safety code compliance 
\item Marketing and ticket printing 
\item cleanliness and organization of Rocky Horror storage and storage of any set pieces that will not fit in the Rocky Horror storage closet.
\end{enumerate}
\end{enumerate}
\item \textbf{Elections.} Elections for Risley Kommittee positions shall be held during the third full week of classes. Procedures for the elections are as follows: 
\begin{enumerate}[A.]
\item Kommittee shall designate an election facilitator. By default, this person shall be the Propagandist, but Kommittee may designate a different election facilitator at its discretion; (rev. Fall 2007)
\item An election information sheet and nomination sheet shall be posted no later than the Thursday of the second full week of classes;
\item Nominations shall be accepted through the Sunday following the second full week of classes; 
\item On the Wednesday preceding the election, ballots will be distributed to all Risley members; 
\item The voting booth location shall be thoroughly publicized prior to the opening of the election; 
\item Ballots will be collected at the voting booth Wednesday through Friday of the third full week of classes; 
\item Ballots shall be available at the voting booth; 
\item The voting booth shall be staffed by volunteers, who, if possible, should not be candidates for office; 
\item Cast ballots shall be placed in a seal container, and names shall be recorded-- separate from ballots--to ensure no person votes more than once; 
\item There shall be no electioneering (i.e., no attempt at verbal persuasion) at the voting booth while voting is going on;
\item Immediately after the voting is completed, two of the elected Kommittee members of the old Kommittee shall tabulate the results. Ballots shall be kept for one week after the election (in the event that election is contested) and then destroyed. (rev. Fall 2007) 
\item In the event of a tie for the last Member-at-Large seat, or any other position in which the last (or only) seat is tied, the affected candidates shall be called together and decide either:
\begin{enumerate}[a.]
\item By unanimous assent that the seat shall go to the victor of a best two-outof-three match of rock-paper-scissors between the two candidates.
\item If either candidate objects, there shall be a runoff.
\end{enumerate}
\item The procedure for a runoff shall be exactly like the general election, except that the ballot shall feature only the tied candidates, and that there shall be one week of notice before the runoff. 
\item In the event of a runoff, the contentious seat shall be filled by a temporary appointee from Risley at large or it shall be left vacant at the discretion of and selected by ¡El Presidente For Life!. 
\item Election results shall be announced at the regularly scheduled Kommittee meeting on the Sunday following the election. Newly elected members shall assume their offices immediately following their acceptance. (rev. Fall 2007) 
\item Victorious candidates may not accept more than one office. 
\item The numeric election results shall be made public after and only after the election.
(moved Fall 2007)
\end{enumerate}
\item \textbf{Special Election.} In the case where a special election is called for (electing the chair, impeaching an officer): 
\begin{enumerate}[A.]
\item Immediately before the election, the candidate’ s shall give speeches presenting their campaign platform. Following the speeches, there shall be a debate between the candidates in which anyone present at Kommittee may ask any question of the candidates, and the candidates shall be able to respond. There shall not be discussion of the candidates without their presence. (rev. Spring 2011) 
\item Votes are tallied by summing the number of ones for each candidate or options; 
\item If a candidate or options obtains a majority of ones and has more votes than any other candidate, that candidate wins; 
\item If no candidate has a majority, the candidate or option with the fewest ones is removed, and all rankings on a ballot below that candidate are advanced one rank; 
\item This process is be repeated until one candidate or option has a majority and the greatest number of votes; that candidate wins; 
\item In the event of a tie between two candidates, the candidate who had the most ones in the initial election shall win; if there is still a tie, a victor shall be selected by rock-paper-scissors. 
\item In the event of a tie between two candidates, the candidate who had the most ones in the initial election shall win; if there is still a tie, a victor shall be selected by rock- paper-scissors, unless either candidate objects. In that case, there will be a re-vote. In the case of a re-vote, members may vote for either of the two candidates, regardless of for whom they voted previously. (Spring 2011)
\end{enumerate}
\item (moved Fall 2007)
\item \textbf{Referendum Procedure}
\begin{enumerate}[A.]
\item If a referendum is called by Kommittee or by petition, the Chair shall solicit brief statements. A sheet shall be posted on the Kommittee bulletin board for at least 24 hours so that people may sign up to write statements. These statements shall be submitted to the membership at least 24 hours prior to the voting.
\item Procedures for ballot collection and counting are the same as for Kommittee elections. (Spring 2011) 
\item Voting on a referendum must take place within five days of the referendum's initiation, unless postponed in accordance with Section VI. H. 
\item In a multiple-choice referendum, those runners-up within five percentage points of the top choice will be sent to a runoff with the top choice. There will not be more than one such runoff. A majority affirmative vote shall be required to approve a proposal overturning a decision reached at a Kommittee meeting. 
\item A referendum shall pass by a simple majority of those voting unless the text of the referendum states otherwise.
\item If a referendum is to be included in the By-Laws, a statement to this effect will be included in it. (moved Spring 2003)
\end{enumerate}
\end{enumerate}
\section*{General Policies}
\begin{enumerate}[1.]
\item \textbf{Parent Organization.} Risley Residential College deigns to recognize as its parent organization the Cornell University Department of Residential Programs. (Spring 1982, name changed Spring 1993, name changed again Spring 2004, and again Fall 2006)

\item \textbf{Applications.} No one will be offered permanent housing in Risley without submitting a written application and being accepted by the New Resident Selection Sub-Committee, or by the RHD if there is no selection subcommittee active.
\item \textbf{Prizes.} No cash, or exorbitant prizes in the opinion of Kommittee, will be given for Risley contests.
\item \textbf{Risley vs. the Other Rule.} Whenever a conflict in use of Risley space or property (e.g. pianos) arises between a Risley member and a non-member, the Risley member shall have priority. This rule shall not apply if the space has been allocated for that time.
\item \textbf{Controversial Posters.} The Risley Kommittee does not consider itself a censor with regard to the content of posters. However, if a poster is seen to excite offense or controversy, the person(s) or organization(s) responsible for that poster shall, when informed of this by a member of the staff or Kommittee, attach their name(s) to the poster(s), or post a separate accompanying sheet for comment. (Fall 1981)
\item \textbf{Religious Policies.} Risley Residential College allows religious ceremonies and displays by its members provided that these ceremonies/displays do not appear to represent the Risley community as a whole.
\item \textbf{Public Policies Rule.} All Restricted Policies and General Policies shall be posted on Kommittee Bulletin board. The residents shall be informed of these policies at the beginning of the year at the General Meeting. (Spring 1993)
\item \textbf{Public Charter Rule.} An up-to-date copy of the Charter shall be posted at all times on Kommittee bulletin board, and kept up to date by the Archivist. (Fall 1985)
\item \textbf{Letters of Marque and Reprisal.} Kommittee may, by a 2/3 vote, issue letters of marque and reprisal. Bearers of these letters are entitled to operate under the colors of the Risley Flag and obtain all the protections thereof.
\item \textbf{House Meeting.} The RisOCs, in consultation from staff and the RHD, will be responsible for House Meeting. (Fall 2004) A. House meeting should be held the Wednesday after the first meeting of the year of Kommittee, but other days close to that are acceptable. (rev. Spring 2012) !
\item \textbf{Socratic Method.} All Kommittee executions shall be carried out by hemlock.\footnote{Blame Joye Harmon (Spring 2007) for this.}
\end{enumerate}
\section*{Risley Positions}
\begin{center}\textit{(These are non-Kommittee positions, elected by Kommittee unless otherwise stated. Holding one of these positions does not guarantee the holder a vote at Kommittee)}
\end{center}
\begin{enumerate}[1.]
\item \textbf{Risley Webmaster.} As needed, Kommittee and the RHD shall designate a Webmaster of the Risley Website. Kommittee may fiscally compensate the Webmaster at its option, with pay determined and allocated at the start of that semester. The duties of the Webmaster shall include:
\begin{enumerate}[A.]
\item Updating and maintaining the website; 
\item Responding to any technical queries about the website; 
\item Implementing any trivial changes to the website architecture; 
\item Sp34k1ng l33t h4x0r.
\end{enumerate}
\item \textbf{Liaison to the Piano Sub-Committee.} One of the Members-at Large shall be appointed liaison to the Piano Sub-Committee, if the piano sub-committee is active.
\item \textbf{Risley Librarian.} As needed, Kommittee shall designate a librarian, bearing in mind the recommendation given by the existing librarian, whose duties shall include: 
\begin{enumerate}[A.] 
\item Creating library rules as the need for them arises and posting those rules on the outside of the library door; (rev. Spring 2009) 
\item Enforcing said library rules;
\item Maintaining an atmosphere of comfort and cleanliness in the library;
\item Performing upkeep of the library collection;
\item Having final approval of all library allocations. (Fall 2002)
\end{enumerate}
\item \textbf{Artist Liason.} The Artist Liason shall generally encourage a continued relationship between previous visiting and resident artists, and Risley. The Liason’s responsibilities include keeping in contact with previous Risley AIRs and GSAs and informing Kommittee of relevant/interesting developments, as well as working with the archivist to document previous AIR/GSA contributions and current works. The Liason is also expected to contact, plan workshops with, and organize interactions with, local artists, as well as propose all allocations and announcements to this end during Kommittee; these responsibilities of the position are meant to encourage involvement between local artists and Risley. (Spring 2012)
\begin{enumerate}[A.]
\item The artist Liason is expected to coordinate with the AIRs, RHDs, RAs, RisOcs, and Kommittee in their duties.
\item The Artist Liason may chair an Artist Liason Subcommittee to work on these responsibilities as necessary.
\end{enumerate}
\item \textbf{Risley Technical Director.} The Technical Director (TD) shall be elected by building referendum at the end of each semester. When replaced, the former TD(s) will carry out their duties until the final date of the Cornell Calendar semester. The incumbent(s) have authority over the new TD(s) until said date. The new TD(s) will have power over the building, but the old TD(s) will have power over the new, acting as mentors.(Spring 2013) This position must be filled at all times. Adequate assessment of skill and knowledge should be involved in the election of this position. The technical director, (hereafter referred to as TD), must be fiscally compensated by Kommittee at a rate of its own choosing. The technical director shall be in charge of providing counsel and direction in all matters of technical production and pertinent equipment, in Risley Hall spaces other than the theater. The technical director is responsible for the design of the technical aspects of all major events but must be aided in the implementation of said events. The duties include, but are not limited to: 
\begin{enumerate}[A.]
\item Overseeing sound, lighting, and video setup of major programs such as M-rave, Rocky, Harry Potter Night, etc. This should be done in concert with the theater's TD when the theater's supplies are in use. 
\item Providing assistance to weekly programs or special events in need of video/ audio support. The Risley TD shall be in possession of the Risley Projector. 
\item Providing counsel in the planning stage of all Risley events in need of anything involving a "tech" setup.
\end{enumerate}
\item \textbf{Archivist.} The archivist shall be tasked with preserving institutional knowledge of use to the building. The archivist shall therefore document Risley and Kommittee procedures, the rationale for major past decisions, and the history of Risley, Charter, and Kommittee, and such other topics as the archivist shall choose or Kommittee shall direct. The archivist shall be given electronic copies of the minutes of Kommittee and subcommittee meetings, and shall make these publicly available in some convenient format. The archivist shall be responsible for maintaining authoritative copies of the Risley Charter and related texts, and making these available to the Risley membership. The archivist may appoint and remove associate archivists, with powers and duties designated by the archivist. The Risley Archivist shall be entitled to such compensation as Kommittee sees fit to award. The archivist shall be appointed by Kommittee bearing in mind the recommendation of the existing archivist. (Spring 2005)
\item \textbf{Compost Master/Mistress.} The Compost master/Mistress shall maintain a student-run compost system in Risley by allocating for necessary supplies, recruiting and training volunteers, and educating the Risley community on proper compost procedure. They shall also work with Risley dining to ensure a continued good relationship over compost issues. (Spring 2012)
\item \textbf{Listerv Guardians.} Five people shall be Listserv Guardians, elected by referendum, in the same manner as members at large. If possible, only one of these people may be New Blood. If possible, at least two of these people must be returners to the position. The position term is one semester. Each of these people must be present at Kommittee to accept their nomination, but it is not necessary that they be endured members of Kommittee either to be elected or to maintain the position. One Listserv Guardian shall be designated “Kommittee Liaison” by the Guardians – that person must attend Kommittee each week. Each of these people shall have equal authority to operate autonomously, but they are a team and need to be uniform in their responses. (Spring 2011)
\begin{enumerate}[A.]
\item The role of the Listserv Guardians shall be to send a neutral form letter to people who send emails to the wrong listserv or to moderate the listservs, depending on whether the listservs are moderated or unmoderated. (Spring 2011) 
\item The RHD is responsible for stepping in in the case of a nasty flame war. And the RAs shall always be able to act in their role of ensuring residents are respectful to each other, regardless of the medium of communication. (Spring 2011)
\end{enumerate}
\item \textbf{Shop Coordinator.} Kommittee shall designate a Shops Coordinator, who has the duties states in section IV. B. and is not an RA (Spring 2011).
\end{enumerate}
\section*{Subcommittees}
\begin{enumerate}[1.]
\item \textbf{New Resident Selection Subcommittee.} Every semester in which Risley has slots available for residency, Kommittee shall form the New Resident Selection Subcommittee as follows. The selection process shall be publicized by the Kommittee Minister of Propaganda. A. Each spring semester, the RHD shall be responsible for soliciting and collecting applications for residence in Risley. They present all applications to the Risley community for review. Applications shall be reviewed as followed: (rev. Spring 2010)
\begin{enumerate}[A.]
\item Residents shall be able to view and rate all applications in the RHD office at a time specified by the RHD. This time period must be at least three hours, and should be split over 2 days, if practical. (rev. Spring 2012)
\item Any Risley Member may rate the applications as he or she sees fit, however, if a member chooses to rate one application, they must rate all applications. Completed rating sheets shall be deposited with the RHD or his or her designee. (rev. Spring 2010) B. Open residency slots for the next semester shall be filled according to the ratings, from greatest to least. C. All accepted applications shall be kept in the Risley Archives.
\end{enumerate}
\item \textbf{Theatre Subcommittee.} The Risley Theatre Subcommittee shall have exclusive use of Risley Theatre. The General Manager shall provide a schedule of theatre use to the Grand Vizier. (Fall 1988, rev. Fall 2007)
\item \textbf{Major Programs.} Before the last meeting of Kommittee the semester before a major Risley program, Kommittee will select people to run the major program. 
\begin{enumerate}[A.]
\item Major programs shall be defined as Masquerave, Handel's Messiah, Spring Faire, A Night At Hogwarts, and Bye Bye Barbeque.
\item The selection process should be publicized at least two weeks in advance, taking into account program interest information from Risley applications collected by the RHD. 
\item Voting will occur as per the election procedure for a special election. 
\item Kommittee may defer this process to the RHD for any major program by majority vote.
\end{enumerate}
\end{enumerate}
\section*{Allocations and Groups}
\begin{enumerate}[1.]
\item \textbf{Fiscal Responsibility.} The following applies to all allocations of Kommittee funds: 
\begin{enumerate}[A.]
\item The person or group who asks for an allocation of Kommittee funds (the allocator) and the Grand Vizier are responsible for keeping track of how much money they have spent. (rev. Fall 2002) 
\item The allocator shall spend no more than the original allocation without an additional allocation from Kommittee. 
\item If more money is spent than was allocated, the allocator is personally responsible for these additional amounts. They may ask Kommittee to increase the allocation, but Kommittee is not obligated to do so. 
\item If Kommittee turns down a request to increase an allocation, the original allocator shall make payment of the additional amount to the Risley treasury. Until such payment is received in full, the original allocator shall neither be allowed to make additional space or money allocations, nor use any previous allocations. 
\item Any group that requests a cumulative sum of money greater than or equal to 2\% of Kommittee's total budget (as calculated only from the program fee for the semester) must give a detailed report of how they have used and plan to use their funds before receiving an allocation or obtaining a further allocation. (Fall 2004)
\end{enumerate}
\item \textbf{Guidelines for the co-sponsorship of events originating outside Risley:}
\begin{enumerate}[A.]
\item Events cosponsored by Risley College should be largely oriented towards the creative or performing arts; 
\item They should allow as much participation of Risley residents, both in the planning and in the event itself, as possible; 
\item The Risley Kommittee should endeavor to encourage and participate in cosponsored events rather than make financial contributions; (rev. Fall 2002) 
\item Monetary co-funding should not be the only participation of Risley; events where no Risley members are likely to show up or participate should not be funded; 
\item Any events co-funded must be publicized within Risley so that residents know about the event; 
\item The Risley Kommittee should encourage allowing events to use space within Risley (to make it easier for Risley members to participate), but should not displace Risley activities. (Fall 1981)
\end{enumerate}
\item \textbf{Music Room Prime Time.} The hours between 5 and 11 p.m. daily shall be defined as "prime time" in the Music Room.
\begin{enumerate}[A.]
\item No groups shall be allowed more than 2 hours of regularly scheduled prime time hours in the Music Room per week. 
\item Exceptions to this limit may be made by a 2/3 vote. (Spring 1984, rev. Fall 2002)
\end{enumerate}
\item \textbf{Maximum Music Room Time Allocation.} In order to be more sensitive to the needs of Risley members, Kommittee shall not allocate more than 26 hours of music room time in one semester to one person or group in a single motion or meeting. (rev. Fall 2002) For example, a group using the music room five hours per week would need to get separate Kommittee approval for every 5 weeks they need in the music room. This would therefore not affect groups who only use the music room 1.5 hours per week (or less). (Spring 1991)
\item \textbf{AIR Welcoming Kommittee.} (Spring 2003, Deleted Spring 2012)
\item \textbf{RisOCs.} The Risley Kommittee shall, from time to time, establish its own orientation committee. This committee shall be called "the RisOCs", and shall assist new residents in moving in and stuff. There shall be a "Lord of the RisOCs", and he or she shall coordinate all RisOC activities. (Spring 2003)
\begin{enumerate}[A.]
\item RisOC applications--comprised of questions posed by the RHD--shall be made available exactly one month before study week. Applications shall be due two weeks later. The RHD shall select the RisOCS. (Fall 2004, rev. Spring 2012) 
\item The Lord of the RisOCs shall be nominated out of the RisOCs after the RisOCs have been accepted that Spring. The RisOCs shall vote on their Lord. The position may be held by a two-headed monster. If possible, the Lord (or at least one of the Lords) must have been a RisOC before. (Fall 2004, rev. Spring 2012) 
\item RisOCs shall be selected each spring for use the following fall. (Fall 2004)
\end{enumerate}
\item \textbf{Implicit Kommittee allocation.} The CLR shall be implicitly allocated for all regular Kommittee meetings. (Fall 2007) 
\end{enumerate}
\section*{Use of Space}
\begin{enumerate}[1.]
\item \textbf{Clean Space Rule.} Space allocations shall be given with the stipulation that the space be kept clean; violations of this stipulation may result in the allocation being revoked.(Fall 1981)
\item \textbf{Murals}
\begin{enumerate}[A.] 
\item \textbf{Mural Approval Policy.} All murals painted in Risley must go through an approval process as follows:
\begin{enumerate}[a.]
\item All murals must be approved by a 2/3 vote of Kommittee.
\item The artist must contact the RHD to fulfill the parent organization's (specified in the By-Laws) requirements for murals. (rev. Spring 2012) 
\end{enumerate}
\item Approval should be sought by the artist before painting the mural, but approval may be received after a mural has been started. If an artist begins a mural before receiving approval, she must notify the RHD immediately to outline her intentions and designs, begin the hallway approval process within two days of starting the mural, and present the design at the next Kommittee meeting. Unfinished or unapproved murals are subject to paint-over responsibilities. (rev. Spring 2008) 
\item \textbf{Existing Murals}
\begin{enumerate}[a.]
\item There will be no additions to existing murals, except by the original artist, or with the consent of the original artist, who must follow the approval procedure stated in the Mural Approval Policy section, for these additions.
\item The only paint which will ever touch an existing mural is paint to refinish the mural, due to damage, age, etc. When refinishing a mural, the paint that is used must as closely as possible resemble the original paint colors.
\item To paint over an existing mural, you must get 30 signatures on a petition to send the issue to referendum. Kommittee may also call for referendum. There is no other way to eradicate a mural. 
\end{enumerate}
\item \textbf{Mural Paint-over Responsibilities.} Unapproved murals and additions to existing murals may be removed at the expense of the artist. (rev. Fall 1990)
\end{enumerate}
\item \textbf{Postering}
\begin{enumerate}[A.]
\item Posters may be placed on walls and doors (excluding the glass on exterior and fire doors) provided they are affixed in a manner that does not cause any damage and is in accordance with the parent organization's (specified in the By-Laws) policies. (rev. Fall 2002, rev. Fall 2007) 
\item Risley poster materials are, for the most part, to be used for advertising Risley programs only. Sole exceptions where materials might be used privately would include: posters to advertise resident-run spontaneous events, or to solicit or offer information on lost/found items. Uses excluded would be in particular those where materials are used to make political statements of any kind (unless, of course, Risley had officially taken a position on some political issue). 
\item Postering outside Risley. When Risley events are advertised outside Risley, Risley will follow postering guidelines set forth by the university. (Spring 1987) 
\item There shall be no postering on Risley murals. (Fall 1981) 
\item Bulletin boards designated by Kommittee for specific purposes shall be used for those purposes only. 
\item Posters must be removed within three (3) days of the final listed date of the program; offenders will be frowned upon. :( (Fall 2002)
\end{enumerate}
\item \textbf{Guest Suite.} Allocation of the rotating guest suite will be handled in the same manner as any other space allocation. Guest Suite Residents staying for longer than one month must be approved by Kommittee. (Spring 1988) An Artist in Residence may serve a maximum of four terms consecutively. This limitation does not apply to non-consecutive terms. (Spring 1993, rev. Spring 2003)
\item \textbf{Tammany.} The Tammany Managers shall have jurisdiction over Tammany from 8pm to 3am Friday and Saturday evenings while Risley is open to undergraduates. Any use of Tammany outside of that time must be approved by the Managers, except when allocated by Kommittee. (Fall 1988)
\item \textbf{Cross Country Gourmet.} Risley Dining shall have the use of Tammany and the Rotunda for Cross Country Gourmet dinners, provided the Risley Kommittee receives notice of the event at least 2 weeks in advance and approves of the allocation. (Fall 1988)
\item \textbf{Misuse of Space.} If a group has been found to misuse Risley space, this group will be issued a written warning by the Grand Vizier to the group contact. Upon evidence for a second misuse of space, the group will forfeit their present allocation and be required to attend the subsequent Kommittee meeting to address the grievance. Upon failure to attend Kommittee, the offending group will be added to the Groups in Poor Standing List in the Outside Groups Appendix. (Spring 2003, rev. Fall 2006)
\item \textbf{Fires in Cowcliffes.} Fires may be burned in the fireplace in Cowcliffes provided that a staff member is present and conscious. If there is a Risley Axe, it will remain, unless in use, in the secret cabinet to the right of the fireplace. (Spring 2003) 
\end{enumerate}
\section*{Equipment}
\begin{enumerate}[1.]
\item \textbf{Risley Flag.} Risley Residential College will adopt an official flag for Risley. Risley Kommittee may opt, by a 2/3 vote, to call for possible flag designs from Risley members. Risley will then approve a design by referendum. Kommittee is responsible for the production and maintenance of any such flag. One physical copy of the flag should be properly displayed, under normal circumstances, somewhere in the common areas of Risley, and another physical copy should be stored by the archivist as a spare. The flag will be displayed at house meetings, and may be allocated for display at certain special events (ex. Slope Day) by a simple majority of Kommittee.
\item \textbf{Risley TV and VCR Policy}
\begin{enumerate}[A.] 
\item The TV room contains one TV maintained by the parent organization specified in the By-Laws. The Risley Kommittee, as an arbiter of TV Room disputes, states the following procedure for the settling of these disputes: "Majority rules, except in the case where a resident/residents is/are watching a program in progress (continuous programming, e.g. The Weather Channel, excluded). There will be a two hour limit for any single program, notwithstanding actual length if it would run longer. Politeness remains key." 
\item The TV Room may be allocated for special events.
\item Any DVDs, VCRs, Camcorders, or other TVs owned by Risley shall be used primarily for programming purposes. They may be allocated for programs by Kommittee. They shall also be available to residents for their own use. However, they may not be reserved for private functions. Risley staff shall oversee the use of this equipment. (Fall 1992)
\end{enumerate}
\item \textbf{Piano Moving.} Pianos must be moved by a professional piano mover when the move involves the disassembly of the piano, up or down stairs, or outside the building. Any other moves can be performed by two or more people. For the purposes of this policy, the Kramer is not a piano. (Fall 1981, rev. Spring 1999)
\item \textbf{The Doughnut Machine Rule.} The Kramer piano shall be designated a non-piano and can be moved non-professionally with Kommittee approval. (Spring 1989)
\item \textbf{The Baby Machine Rule.} The photocopier adjacent to the Soda machines in the basement shall be designated a non-photocopier and a Baby Machine. (Fall 2002)
\item \textbf{Don't Eat it Clause.} The Risley Charter in digital form or paper shall not be ingested. (Spring 2007) \footnote{Blame Kolb Ettenger (Spring 2007) for this.}
\end{enumerate}
\section*{Outside Groups Appendix}
\begin{enumerate}[1.]
\item \textbf{Good Standing.} The following groups are to be considered to have "Good Standing" for all purposes of the Risley Kommittee and Charter: 
\begin{enumerate}[a.]
\item The Society for Creative Anachronism (or, "SCA") 
\item Denton Drama Troupe
\item Out Loud Chorus
\end{enumerate}
\item \textbf{Poor Standing.} The following groups are considered to have "Poor Standing" for all purposes of the Risley Kommittee and Charter: 
\begin{enumerate}[a.]
\item None at this time.
\end{enumerate}
\item \textbf{SCA Provisions.} The Society for Creative Anachronism shall have the use of three closets in the basement for armor storage. In addition, they shall have use of the Great Hall for fighting practice on Sunday afternoons from noon to 5pm and for arts meetings on Tuesday evenings from 8:30pm to 11pm. The SCA agrees to be flexible about this schedule, and to give way to their allocations if requested by Kommittee. Individual SCA members who are not already Risley members will be asked to pay shop fees to the particular shops which they use. (Fall 1988, moved Fall 2004)
\end{enumerate}
\end{document}
